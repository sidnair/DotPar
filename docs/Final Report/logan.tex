Project Manager - Logan
<<<<<<< HEAD
\label{sec:project_plan}

=======

\section{Process}
The initial planning of our language took place during in-person meetings where we were able to discuss and critique ideas quickly and effectively. We explored the differences between and application and a language and as a result were able to eliminate many of our ideas. We eventually narrowed our scope down to focusing on parallelization. After getting constructive feedback on our white paper we realized that our scope was too broad for the amount of time we had to complete the project. We then decided to narrow this scope down to parallelization around data manipulation, specifically in arrays. 

Through out the course of the project, we have held weekly in-person meets on Fridays. This was an opportunity for the group to check in and make sure everyone was on the same page. After an hour or so everyone would break up and work on their respective tasks for that week while still allowing for collaboration. 

GitHub became an invaluable resource to the group as a shared repository and version control. In addition, we used the GitHub issues and milestones features to set goals for when certain parts should be completed.  Issues allowed us to post tasks and problems that can be assigned to an individual or multiple people. It then easily allows us to see what is left to do and what has been accomplished already. This allowed us to have a scheduling system that was directly tied to our repository and stay on track.

We created tests at each stage of development to ensure that our work was correct and that we could proceed without causing problems in the next stage.

\section{Roles and Responsibilities}
The assigned roles for the group are:
>>>>>>> Final Report updates
\begin{itemize}
\item Logan Donovan – Project Manager
\item Sid Nair – Language Guru
\item Justin Hines – System Architect
\item Nathan Hwang - System Integrator
\item Andrew Hitti	- System Tester
\end{itemize}

While all of the positions come with specific tasks assigned to them, each member of the group was able to contribute to the high-level design of the language.  From there each person became responsible for a specific portion of the project and then received help from other group member as needed. Nathan worked on the parser and abstract syntax tree creation. Andrew and Justin built the semantic analysis and code generation. Sid was responsible for parallelization in Scala. Logan wrote the sample programs and served as the primary coordinator between individuals to help make sure everyone was working together. In addition she wrote a good deal of the original type checking system, which was later replaced with a more efficient model and no longer, appears in our code. Despite these roles, all groups' members participated in a great deal of cross-collaboration.

\section{Style Sheet}
\begin{itemize}
\item Two spaces as indentation 
\item File names and values use \_ underscores
\item Functions use lowercaseUppercase word concatentaion 
\end{itemize}

We used an OCAML Style guide that can be found at:
\verb=http://caml.inria.fr/resources/doc/guides/guidelines.en.html=

\section{Timeline}
\begin{description}
\item{March 21, 2012} Language Reference Manual and Language Tutorial Due

\end{description}
\section{Project Log}

\begin{itemize}

\item Include the implementation style sheet used by the team.
\item Show the timeline of what was done and when.
\item Include your project log.
\end{itemize}
